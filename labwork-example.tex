% arara: xelatex: { shell: true, interaction: nonstopmode } if changed('tex') || changed(toFile('fefudoc.cls'))
%arara: biber if changed('bib') || changed('bcf') || changed('bbl')
%arara: xelatex: { synctex: true, shell: true } if changed('aux')
% !TeX document-id = {2e2f2d46-ef9b-47ad-990b-8fdc6dcf34fb}
% !TeX program = xelatex
% !TeX encoding = UTF-8
% !TeX spellcheck = ru_RU
% !TeX TXS-program:compile=txs:///xelatex/[--shell-escape]|txs:///view-pdf
% !TeX TXS-program:bibliography = txs:///biber

% Тип документа:
\documentclass[labwork]{fefudoc}

\usepackage{mathtools, amsfonts, amssymb} %основные математикие операторы и шрифты
\usepackage{graphicx} %вставка рисунков из файлов
\usepackage{listings} %исходный код
\lstset{ %общие параметры расширения listings
	basicstyle=\ttfamily, %основной стиль текста кода
	keywordstyle=\bfseries, %стиль ключевых слов
	commentstyle=\itshape, %стиль комментариев
	showstringspaces=false, %не подчеркивать пробелы в строках
	frame=single, %обвести рамкой
	tabsize=2 %размер табуляционного сдвига
}
\usepackage{threeparttable}

\location{Владивосток}
\Institute{Политехнический институт (Школа)}
\Department{Департамент электроники, телекоммуникации и приборостроения}
\specialty{11.03.02} %коды специальностей приведены в Общероссийском классификаторе 009-2016,
                     % а также в файлах details/barchelorspecialties.def (бакалавры)
					 % и в details/masterspecialties.def (магистранты)
\profile{Системы радиосвязи и радиодоступа}
\discipline{Параллельное программирование}

%Автор:
\author{Б3121-11.03.02сррд}{Герасименко Эдуард}
%Преподаватель:
\teacher{кандидат технических наук}{}{доцент департамента ЭТиП}{Чусов Андрей Александрович}

%Название работы
\title{Изучение программной среды \LaTeX{} для написания литературных работ}

\year=2024

\begin{document}
\frontpage
\tableofcontents

\section{Эдуард Герасименко}
Говорун — довольно ленивая птица. Когда не говорит и не летает, любит дремать. Если много разговаривает, может охрипнуть. Когда говорун много говорит, особенно на языке космических пиратов, он хрипнет и скрипит, как несмазанная телега. Чтобы вернуть говоруну голос, его нужно напоить горячим молоком, лучше с содой.

Данные о речевых способностях говоруна несколько противоречивы: отмечается, что говорун способен только повторять фразы, услышанные от кого-либо (также имитируя голос и интонации) — но в нескольких эпизодах говорун произносит вполне осмысленные предложения, которые едва ли могут быть повторением чьих-то чужих слов.

Кормить говорунов можно обычным белым хлебом с молоком. Для полезности им дают шиповниковый сироп. Также они любят сахар.

Впрочем, во время межзвёздных перелётов говоруны могут долго обходиться без пищи; в случае же сильного голода говорун может откормиться даже машинным маслом (у роботов с планеты Шелезяка не было для говоруна другой еды).

\section{Александр Кулдошин}
\section{Полина Сараева}
\section{Дмитрий Зинченко}

\end{document}

