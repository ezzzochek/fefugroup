% arara: xelatex: { shell: true, interaction: nonstopmode } if changed('tex') || changed(toFile('fefudoc.cls'))
%arara: biber if changed('bib') || changed('bcf') || changed('bbl')
%arara: xelatex: { synctex: true, shell: true } if changed('aux')
% !TeX document-id = {2e2f2d46-ef9b-47ad-990b-8fdc6dcf34fb}
% !TeX program = xelatex
% !TeX encoding = UTF-8
% !TeX spellcheck = ru_RU
% !TeX TXS-program:compile=txs:///xelatex/[--shell-escape]|txs:///view-pdf
% !TeX TXS-program:bibliography = txs:///biber

% Тип документа:
\documentclass[labwork]{fefudoc}

\usepackage{mathtools, amsfonts, amssymb} %основные математикие операторы и шрифты
\usepackage{graphicx} %вставка рисунков из файлов
\usepackage{listings} %исходный код
\lstset{ %общие параметры расширения listings
	basicstyle=\ttfamily, %основной стиль текста кода
	keywordstyle=\bfseries, %стиль ключевых слов
	commentstyle=\itshape, %стиль комментариев
	showstringspaces=false, %не подчеркивать пробелы в строках
	frame=single, %обвести рамкой
	tabsize=2 %размер табуляционного сдвига
}
\usepackage{threeparttable}

\location{Владивосток}
\Institute{Политехнический институт (Школа)}
\Department{Департамент электроники, телекоммуникации и приборостроения}
\specialty{11.03.02} %коды специальностей приведены в Общероссийском классификаторе 009-2016,
                     % а также в файлах details/barchelorspecialties.def (бакалавры)
					 % и в details/masterspecialties.def (магистранты)
\profile{Системы радиосвязи и радиодоступа}
\discipline{Параллельное программирование}

%Автор:
\author{Б3121-11.03.02сррд}{Герасименко Эдуард}
%Преподаватель:
\teacher{кандидат технических наук}{}{доцент департамента ЭТиП}{Чусов Андрей Александрович}

%Название работы
\title{Изучение программной среды \LaTeX{} для написания литературных работ}

\year=2024

\begin{document}
\frontpage
\tableofcontents

\section{Эдуард Герасименко}
Я закончил школу №23 в 2003~г. с углубленным изучением физики и математики. Моими любимыми предметами были физика, математика, а также технология. С детства меня интересовали средства и методы автоматизации технических процессов, способы управления и передачи управляющих сигналов.

Также в свободное время я люблю заниматься созданием видео и музыкой. Поэтому я поступил на нашу специальность в надежде получить знания и умения использовать современные технологии и средства построения инфокоммуникационных систем, научиться думать рационально и самостоятельно принимать решения о том, как подходить к проблемам, возникающим в этих областях знаний.

В будущем я надеюсь работать по профессии в таких компаниях, как Ростелеком или МТС, или решать задачи проектирования сетей связи в условиях специальных требований к надежности, отказо- и помехоустойчивости, информационной безопасности.


\section{Александр Кулдошин}
\section{Кирилл Березин}
\section{Денис Скоробогатов}
Ты хотел изменить свою жизнь, но не мог этого сделать сам. Я — то, кем ты хотел бы быть. Я выгляжу так, как ты мечтаешь выглядеть. 
Я трахаюсь так, как ты мечтаешь xx@#$pgся. Я умён, талантлив и, самое главное, свободен от всего, что сковывает тебя.
\section{Максим Алексеев}
\section{Полина Сараева}
\section{Дмитрий Зинченко}

\end{document}

