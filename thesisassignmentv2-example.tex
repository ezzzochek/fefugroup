% arara: xelatex: { shell: true, interaction: nonstopmode } if changed('tex') || changed(toFile('fefudoc.cls'))
%arara: biber if changed('bib') || changed('bcf') || changed('bbl')
%arara: xelatex: { synctex: true, shell: true } if changed('aux')
% !TeX document-id = {0a346c97-4fec-4487-b840-d4acdead04bb}
% !TeX program = xelatex
% !TeX encoding = UTF-8
% !TeX spellcheck = ru_RU
% !TeX TXS-program:compile=txs:///xelatex/[--shell-escape]|txs:///view-pdf
% !TeX TXS-program:bibliography = txs:///biber
\documentclass[ {{barchelor thesis assignment}} ]{fefudoc}
\usepackage{url}

\Student{Андреев Борис Викторович}
\Tutor{Борисов Виктор Геннадиевич}
\Group{Б3119-11.03.02сррд}
\Specialty{11.03.02}
\AssignmentDate{\today}
%\AssignmentAcceptanceDate{24.06.2023}

\begin{document}
\begin{Assignment}
\AssignmentTheme{Реализация и анализ табличных методов приближения цифровых сигналов}
\AssignmentDueTime{}
\AssignmentCoverage{1. Источники научно-технической литературы в базах данных реферируемых журналов. 2. В. М. Хачумов, Вычисление математических функций на основе разряднопараллельных схем, ИТиВС, 2016, выпуск 3, 26-44 3. Godbole B.B., Nikam R.H. FPGA implementation of CORDIC algorithm used in DDS based modulators. International Journal of Advanced Research in Computer and Communication Engineering, Vol.4, Issue 1, January 2015, pp.94-97.}
%\AssignmentSources{Техническое задание. Учебные пособия, статьи.}
\AssignmentTopics{
1. Современные методы табличного приближения аналитических функций в машинных вычислениях.
2. Использование арифметики с плавающей точкой для численного приближение и анализ ошибок аппроксимации, обусловленный арифметикой.
3. Метод реализации приближения аналитических функций с помощью табличных предрасчетов.
4. Разработка экспериментального образца и экспериментальное исследование точности приближения и оперативности.
}
\AssignmentPosters{
1. Блок-схема реализации алгоритма приближения синусоидального и косинусоидального сигналов с помощью таблиц CORDIC.
2. Оценка эффективности метода CORDIC для функций SIN и COS.
3. Сравнение эффективности методов CORDIC и аппроксимации  при помощи рядов Тейлора.
4. Оценка эффективности табличного метода аппроксимации функции LOG2(M) для 6000 битовой точности. 
}
\AssignmentConsultants{}
\AssignmentSources{
1. Rui Xu, Zhanpeng Jiang, Hai Huang, Changchun Dong. An Optimization of CORDIC Algorithm and FPGA Implementation. – International Journal of Hybrid Information Technology Vol.8, No.6 (2015), pp.217-228. – URL: \url{http://www.sersc.org/journals/IJHIT/vol8_no6_2015/21.pdf}.
2. Muller, J. M., Elementary functions : algorithms and implementation / Jean-Michel Muller.– 2nd ed., 2005.
3. Shrugal Varde, Dr. Nisha Sarwade, Richa Upadhyay. Hardware Implementation Of Hyperbolic Tan Using Cordic On FPGA International Journal of Engineering Research and Applications (IJERA), Vol.3, Issue 2, March-April 2013, pp.696-699.
}
\end{Assignment}

\end{document}

