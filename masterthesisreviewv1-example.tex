% arara: xelatex: { shell: true, interaction: nonstopmode } if changed('tex') || changed(toFile('fefudoc.cls'))
%arara: biber if changed('bib') || changed('bcf') || changed('bbl')
%arara: xelatex: { synctex: true, shell: true } if changed('aux')
% !TeX document-id = {017b6220-b2ca-4105-8812-f10ca5a1a88d}
% !TeX program = xelatex
% !TeX encoding = UTF-8
% !TeX spellcheck = ru_RU
% !TeX TXS-program:compile=txs:///xelatex/[--shell-escape]|txs:///view-pdf
% !TeX TXS-program:bibliography = txs:///biber
\documentclass[ {{master thesis review}} ]{fefudoc}

\student{Андреев Борис Викторович}
\subject{Анализ уязвимостей беспроводных сетей Wi-Fi и методы борьбы с ними}
\tutor{кандидат технических наук}{}{доцент департамента ЭТиП}{Борисов Виктор Геннадиевич}
\specialty{11.04.02}
\profile{Системы радиосвязи и радиодоступа}


\begin{document}
\Actuality
%Раскрывается основное значение исследуемой в выпускной работе темы, ее актуальность (для кого, чего), характер (прикладной, теоретический и т.д.). Отмечается, почему выпускник выбрал (или ему доверили) эту тему для разработки, либо отмечается, что тема – инициативная.
В работе студентом исследуются методы реализации информационной безопасности в беспроводных сетях Wi-Fi. Приведен анализ существующих высокоуровневых протоколов, включая WPA, WPA2 и WPA3, показаны используемые в них криптографические примитивы и наиболее основные методы реализации атак. Актуальность темы обусловлена необходимостью обеспечения информационной безопасности в постоянно растущих информационных сетях, включая Wi-Fi, и их использованием в местах большого колчества недоверяющих друг другу сторон, таких как места публичного пользования сетью, и, как следствие растущей простотой внедрения в каналы связи третьего лица для осуществления им атаки, а также вычислительных возможностей для реализации этой атаки. Характер работы - прикладной, выбор магистрантом темы обусловлен научным интересом.

\WorkingProcess
%Что и в каком объеме сделано обучающимся в процессе работы, насколько он (она) освоили методы научного (практического) решения поставленных задач, уровень их исполнения. Отмечается ответственность, ритмичность работы и т.п. Особо подчёркивается степень самостоятельности обучающегося в выполнении работы. Указывается (если имеется), что результаты работы были опубликованы и/или представлены на конференции (неделе науки и т.д.), по результатам чего работа Фамилия и инициалы обучающегося была отмечена там-то.
Данил, поступив в магистратуру ДВФУ, приступил к работе, не изучав, по его словам, ранее в бакалавриате дисциплины, связанные с обеспечением информационной безопасности инфокоммуникационных систем и сетей, поэтому проявил некоторую самостоятельность и настойчивость в изучении основных упущенных аспектов применения и использования информационной безопасности, включая безопасность сетей Wi-Fi. Поставленные задачи выполнены лишь частично, но в мере ожидаемой от студента, выполняющего работу в новой для себя области знаний. Результаты работы магистранта представлены в материалах пяти научных конференций:
1) международная научно-практическая конференция \quot{Цифровизация: новые тренды и опыт внедрения}, г. Саратов, 2023~г;
2) всероссийская научно-практическая конференция \quot{Российская наука в фокусе перемен}, г. Уфа, 2023~г;
3) национальная (всероссийская) научно-практическая конференция \quot{Общество "--- Наука "--- Инновации}, г. Волгоград, 2023~г. Имеет диплом II степени в номинации "Технические науки" конкурса \quot{Лучшая исследовательская работа 2022}.

\Drawbacks
Работа выполнена поверхностно, большая часть пояснительной записки посвящена обзору существующих технологий и методов реализации, что обусловлено недостаточностью знаний в области информационной безопасности сетей связи, а также математических методов обработки сигналов и данных. Описание экспериментов, выполненных диссертантом, также ограничено перечислением существующих инструментов для осуществления атак и опытом применения одного из них для наблюдения за траффиком, что недостаточно полно отражает суть проводимого в работе исследования. 
%Указываются замечания (если имеются), которые отразились на качестве выполнения выпускной работы: недостаточность знаний, поверхностность, неритмичность и т.п.

\Defendable[да]

\Mark{4}

\Recomendations
%\RecomendedForScience
\RecomendedForPublication
%\RecomendedForFurtherEducation

\end{document}

