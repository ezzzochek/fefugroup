% arara: xelatex: { shell: true, interaction: nonstopmode } if changed('tex') || changed(toFile('fefudoc.cls'))
%arara: biber if changed('bib') || changed('bcf') || changed('bbl')
%arara: xelatex: { synctex: true, shell: true } if changed('aux')
% !TeX document-id = {551fad21-5fdb-4fab-ac20-3b1fc1515c8f}
% !TeX program = xelatex
% !TeX encoding = UTF-8
% !TeX spellcheck = ru_RU
% !TeX TXS-program:compile=txs:///xelatex/[--shell-escape]|txs:///view-pdf
% !TeX TXS-program:bibliography = txs:///biber
\documentclass[ {{barchelor thesis review}} ]{fefudoc}

\student{Андреев Борис Викторович}
\subject{Разработка высокопроизводительных алгоритмов и оценка оперативности реализации математических методов симметричной и асимметричной криптографической обработки данных}
\tutor{кандидат технических наук}{}{доцент департамента ЭТиП}{Борисов Виктор Геннадиевич}
\specialty{11.03.02}
\profile{Системы радиосвязи и радиодоступа}


\begin{document}
\Actuality
%Раскрывается основное значение исследуемой в выпускной работе темы, ее актуальность (для кого, чего), характер (прикладной, теоретический и т.д.). Отмечается, почему выпускник выбрал (или ему доверили) эту тему для разработки, либо отмечается, что тема – инициативная.
Актуальность работы обусловлена растущими требованиями к эффективности обработки постоянно возрастающего объема данных, передаваемых по каналам связи, инфокоммуникационным сетям или хранимых в накопителях данных, а также, другой стороны, ростом требований к стойкости алгоритмов информационной безопасности и, следовательно, ресурсоёмкости нижележащих алгоритмических и математических примитивов. Рост сценариев применения таких примитивов обуславливает использование составных, и вычислительно сложных, алгоритмов и инфокоммуникационных протоколов. Поэтому исследования вычислительных методов криптографии, и, в частности, методов алгоритмической реализации методов конечных алгебр для эффективного решения задач информационной безопасности, являются актуальными и востребованными, а работа студента носит научно-практический характер. В процессе обучения в бакалавриате студент неоднократно проявлял интерес к вопросам цифровой алгоритмической обработки сигналов и данных, при этом показывая способность решать сложные и неординарные задачи, чем обусловлен его выбор темы работы.

\WorkingProcess
%Что и в каком объеме сделано обучающимся в процессе работы, насколько он (она) освоили методы научного (практического) решения поставленных задач, уровень их исполнения. Отмечается ответственность, ритмичность работы и т.п. Особо подчёркивается степень самостоятельности обучающегося в выполнении работы. Указывается (если имеется), что результаты работы были опубликованы и/или представлены на конференции (неделе науки и т.д.), по результатам чего работа Фамилия и инициалы обучающегося была отмечена там-то.
Студент, в процессе работы проявивший значительную самостоятельность и способность собирать, критически анализировать, интерпретировать, компоновать и систематизировать новые знания, выполнил основные поставленные задачи, удовлетворил требованиям, предъявляемым к квалификационным работам выпускника университета по направлению подготовки 11.03.02, а также требования задания на ВКР. Это, по моему мнению, демонстрирует достаточность умения выполнять поиск научного и практического решения поставленных задач. Результаты работы представлены и опубликованы в материалах конференции «Молодежь и научно-технический прогресс» за 2022 и 2023 годы. Оригинальность работы составляет 94\%.

%Drawbacks
%Указываются замечания (если имеются), которые отразились на качестве выполнения выпускной работы: недостаточность знаний, поверхностность, неритмичность и т.п.

\Defendable[да]

\Mark{5}

\Recomendations
\RecomendedForScience
\RecomendedForPublication
\RecomendedForFurtherEducation

\end{document}

