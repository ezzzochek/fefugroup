\chapter*{Аннотация}
Тема ВКР: "\thetitle".

Объем –-- \totalpagecount{}, \totalfigurecount{}, \totaltablecount{}, список литературы: \totalcitationcount{}.

При выполнении работы использовалось программное обеспечение Microsoft Visual Studio, Notepad++, Intel® Software Development Emulator.

Структурно ВКР в себя включает: введение, аналитическую часть, проектную часть, реализацию, а также заключение и список литературы.

Во введении приводятся актуальность проблемы, постановка цели и постановка задач, необходимых для её достижения.

В аналитической части рассматриваются основные типы алгоритмов быстрого преобразования Фурье.

В проектной части выполняется проектирование и разработка скалярной формы представления алгоритма теоретико-числового преобразования.

В реализации выполняется векторное представление обоснованного алгоритма с помощью набора инструкций AVX-512.

В экспериментальной части ставятся численные эксперименты, по оценке эффективности разработанного алгоритма.

В заключении приводятся результаты проделанной работы.

Ключевые слова: Дискретное преобразование Фурье, быстрое преобразование Фурье, AVX-512, векторные инструкции.

