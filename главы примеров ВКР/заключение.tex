\chapter*{Заключение}
Различные алгоритмы быстрого и дискретного преобразований Фурье на настоящий момент являются наиболее эффективными средствами обработки данных.
Это реализовано за счёт быстрого выполнения свёрточных операций над цифровыми данными.
При этом теорема о свёртке показывает, что сложность свертки во временной области снижается до линейной сложности эквивалентного поточечного умножения в частотной области.
В данной работе использовалось представление теоремы о свёртке в виде теоретико-числового преобразования Фурье над элементами целочисленных колец.
Это обеспечивает реализацию одного из наиболее быстрых на сегодняшний день алгоритмов длинного целочисленного умножения.

Подобные алгоритмы используются в различных приложениях высокоточного вычисления, в том числе при выполнении операций с плавающей точкой, в асимметричной криптографии и в помехоустойчивом кодировании.

В ходе выполнения дипломной работы был реализован алгоритм Шёнхаге-Штрассена в векторной форме.
В данной форме алгоритм используется для длинного умножения целочисленных данных.

Был проведён анализ научно-технической литературы по проблеме применения и высокопроизводительной реализации алгоритмов преобразования Фурье над целыми числами, арифметики произвольных конечных целочисленных колец, применения и эффективной реализации сверточных операций.

Были спроектированы алгоритмы дискретного и быстрого теоретико-числовых преобразований.

Была выполнена векторная реализация свертки над целочисленными кольцами и длинного умножения, с использованием векторного процессора с поддержкой набора векторных инструкций AVX-512.

