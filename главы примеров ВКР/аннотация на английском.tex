%Только для магистрантов
\chapter*{Abstract}
Thesis title is "Vulnerability analysis of wireless Wi-Fi networks and methods of their mitigation".

The work contains \totalpagecount*{} pages, \totalfigurecount*{} figures, \totaltablecount*{} tables; the list of references includes \totalcitationcount*{} entities.

The following tools were used in the process of preparing the thesis: Microsoft Visual Studio, Notepad++, Intel® Software Development Emulator.

Structurally, the thesis consists of an introduction, an analytical part, a design part, an implementation as well as conclusion and a list of references.

The introduction contains the problem statement, formulation of the main purpose of the work and its decomposition onto sub-goals achieved as described within the paper.

The analytical part describes main types of fast Fourier transformation.

The design part contains description of the design and development of the scalar representation of the number-theoretical transformation.

The implementation part describes a developed vectorized representation and a mathematical description of the algorithm for the number-theoretical transformation based on the AVX-512 instruction set.

The experimental part describes numerical experiments conducted in order to assess efficiency of the developed algorithm.

The conclusion outlines the main results of the work.

Keywords: discrete Fourier transform, fast Fourier transform, AVX-512, vector instructions.

