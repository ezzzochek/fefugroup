\chapter{Таблица из РПД}
\renewcommand\theadfont{\footnotesize\bfseries}
\footnotesize
\begin{longtable}{|l|p{3cm}|p{2.5cm}|p{3.3cm}|p{2.2cm}|p{2.25cm}|}
\caption{Планируемые результаты обучения по дисциплине}\label{Таблица РПД}
\endfirsthead
\caption{продолжение}
\endhead
\hline
\multirow{3}{*}{\thead[l]{№\\п/п}} &
\multirow{4}{*}
 {\thead[l]{Контролируемые\\модули/разделы/\\темы дисциплины}} &
\multirow{5}{*}
 {\thead[l]{Код\\идентификатора\\достижения\\компетенции}} &
\multirow{2}{*}{\thead[l]{Результаты обучения}} &
\multicolumn{2}{c|}{\thead[l]
 {Оценочные средства --- \\ наименование}} \\
\cline{5-6}                           &
                                      &
                                      &
                                      &
\thead[l]{текущий\\контроль}          &
\thead[l]{промежуточная\\аттестация} \\
                                      &
                                      &
                                      &
                                      &
                                      &
\\
\hline
1 & Методология автоматизированного проектирования
  & ОПК-1.1 Систематизирует положения, законы и методы в области математики, естественных и технических наук для решения задач управления
  & Знает актуальные методы автоматизированного проектирования технических систем и фундаментальных принципов автоматизированного проектирования.
  & Собеседование (УО-1); дискуссия (УО-4); конспект (ПР-7).
  & Вопросы к зачету 1, 4, 7, 10. \\
\cline{4-6}
  &
  &
  & Умеет применять методы автоматизированного проектирования технических систем, формулировать и обосновывать адекватные
    требования к системам автоматизированного проектирования.
  & Собеседование (УО-1); дискуссия (УО-4); конспект (ПР-7).
  & Вопросы к зачету 1, 4, 7, 10. \\
\cline{4-6}
  &
  &
  & Владеет навыками использования методов проектирования, моделирования и анализа технических систем с помощью систем автоматизированного проектирования.
  & Собеседование (УО-1); дискуссия (УО-4); конспект (ПР-7).
  & Вопросы к зачету 1, 4, 7, 10. \\
\cline{3-6}
  &
  & ОПК-1.2 Выявляет сущность проблем управления
  & Знает методы компьютерной реализации систем автоматизированного проектирования, принципов использования распространенных систем компьютерного проектирования и моделирования технических систем.
  & Собеседование (УО-1); дискуссия (УО-4); конспект (ПР-7).
  & Вопросы к зачету 2, 5, 8. \\
\cline{4-6}
  &
  &
  & Умеет самостоятельно решать задачи проектирования, моделирования, синтеза и анализа компьютерных моделей
    технических систем, выбирать и развертывать адекватные компоненты комплекса средств компьютерного
    автоматизированного проектирования и приводить формальное обоснование принятых решений документально.
  & Собеседование (УО-1); дискуссия (УО-4); конспект (ПР-7).
  & Вопросы к зачету 2, 5, 8. \\
\cline{4-6}
  &
  &
  & Владеет навыками решения задач профессиональной деятельности, проектирования, моделирования и анализа технических систем и устройств, с помощью средств компьютерного автоматизированного проектирования.
  & Собеседование (УО-1); дискуссия (УО-4); конспект (ПР-7).
  & Вопросы к зачету 2, 5, 8. \\
\cline{3-6}
  &
  & ОПК-2.1 Формулирует задачи управления в технических системах
  & Знает принципы развертывания и использования бытовых и промышленных систем компьютерного дизайна и моделирования технических систем, ожидаемые значения критериев эффективности применяемых в профессиональной деятельности САПР.
  & Собеседование (УО-1); дискуссия (УО-4); конспект (ПР-7).
  & Вопросы к зачету 3, 6, 9.\\
\cline{4-6}
  &
  &
  & Умеет самостоятельно решать задачи проектирования, моделирования, синтеза и анализа компьютерных моделей технических систем, выбирать и развертывать адекватные компоненты комплекса средств компьютерного автоматизированного проектирования в соответствии с принятыми требованиями к эффективности и приводить формальное обоснование принятых решений документально в соответствии с отраслевыми, государственными и международными спецификациями.
  & Собеседование (УО-1); дискуссия (УО-4); конспект (ПР-7).
  & Вопросы к зачету 3, 6, 9.\\
\cline{4-6}
  &
  &
  & Владеет навыками обоснования, синтеза и анализа проектных решений в области проектирования технических систем с помощью средств компьютерного автоматизированного проектирования.
  & Собеседование (УО-1); дискуссия (УО-4); конспект (ПР-7).
  & Вопросы к зачету 3, 6, 9.\\
\cline{3-6}
  &
  & ОПК-4.1 Разрабатывает критерии систем управления в области инновационной деятельности & Знает требования к документации на технические сети и системы, а также на средства сопровождения автоматизированного проектирования, методы поиска  регламентирующих государственных и международных стандартов и спецификаций; методы автоматизированной генерации, хранения, сопровождения документации с помощью существующих систем автоматизированного проектирования и программных средств автоматизированной генерации, обработки, хранения и доступа к документации.
  & Собеседование (УО-1); дискуссия (УО-4); конспект (ПР-7).
  & Вопросы к зачету 1, 4, 7, 10.\\
\cline{4-6}
  &
  &
  & Умеет использовать средства автоматизированной генерации, обработки, хранения и доступа к документации; составлять нормативную документацию к системам автоматизированного проектирования технических систем. & Собеседование (УО-1); дискуссия (УО-4); конспект (ПР-7). & Вопросы к зачету 1, 4, 7, 10.\\
\cline{4-6}
  &
  &
  & Владеет навыками использования средств автоматизированной генерации, обработки, хранения, сопровождения и доступа к документации; составления нормативной документации к системам автоматизированного проектирования технических систем в соответствии с отраслевыми, государственными и международными спецификациями.
  & Собеседование (УО-1); дискуссия (УО-4); конспект (ПР-7).
  & Вопросы к зачету 1, 4, 7, 10.\\
\cline{3-6}
  &
  & ОПК-4.2 Систематизирует современные математические методы для разработки критериев систем управления в области инновационной деятельности & Знает методы обоснования критериев эффективности технической системы, основные показатели функциональной и нефункциональной эффективности на примерах ГОСТ ИСО Р 9000-2015, ГОСТ Р 57330-2016 и ГОСТ Р ИСО/МЭК 25010-2015, знает аспекты реализации имитационного моделирования технических систем и его ограничения.
  & Собеседование (УО-1); дискуссия (УО-4); конспект (ПР-7).
  & Вопросы к зачету 2, 5, 8.\\
\cline{4-6}
  &
  &
  & Умеет прогнозно оценивать эффективность системы, используя аналитические и численные модели технической системы, включая методы вероятностной оценки загрузки системы и ее функционирования в нестационарном режиме.
  & Собеседование (УО-1); дискуссия (УО-4); конспект (ПР-7).
  & Вопросы к зачету 2, 5, 8.\\
\cline{4-6}
  &
  &
  & Владеет методами аналитического и численного обоснования критерия эффективности системы, а также оптимизации путем выполнения, в том числе автоматизированного, функционального анализа критерия эффективности применительно к основной и вспомогательной функциям системы.
  & Собеседование (УО-1); дискуссия (УО-4); конспект (ПР-7).
  & Вопросы к зачету 2, 5, 8.\\
\cline{3-6}
  &
  & ОПК-4.3 Вырабатывает и реализует управленческие решения в области инновационной деятельности
  & Знает требования к документации на технические сети и системы, а также на технические средства сопровождения автоматизированного проектирования, регламентирующие государственные и международные стандарты; методы автоматизированной генерации, хранения, сопровождения документации с помощью существующих систем автоматизированного проектирования и программных средств автоматизированной генерации, обработки, хранения и доступа к документации.
  & Собеседование (УО-1); дискуссия (УО-4); конспект (ПР-7).
  & Вопросы к зачету 3, 6, 9.\\
\cline{4-6}
  &
  &
  & Умеет применять средства поиска нормативной документации, воспринимать содержание нормативных спецификаций. Умеет применять средства разметки для автоматизированной генерации, представления, хранения, предоставления доступа и сопровождения нормативной документации к техническим программным и аппаратным решениям при разработке и анализе технических систем.
  & Собеседование (УО-1); дискуссия (УО-4); конспект (ПР-7).
  & Вопросы к зачету 3, 6, 9.\\
\cline{4-6}
  &
  &
  & Владеет навыками использования средств автоматизированной генерации, обработки, хранения, сопровождения и доступа к документации к техническим системам, реализуемым и применяемым при разработке и поддержке телекоммуникационных систем и сетей; составления нормативной документации к системам автоматизированного проектирования технических систем в соответствии с отраслевыми, государственными и международными спецификациями.
  & Собеседование (УО-1); дискуссия (УО-4); конспект (ПР-7).
  & Вопросы к зачету 3, 6, 9.\\
\hline
2 & Техническое обеспечение систем автоматизированного проектирования
  & ОПК-2.1 Формулирует задачи управления в технических системах
  & Знает принципы развертывания и использования бытовых и промышленных систем компьютерного дизайна и моделирования технических систем, ожидаемые значения критериев эффективности применяемых в профессиональной деятельности САПР.
  & Собеседование (УО-1); дискуссия (УО-4); конспект (ПР-7).
  & Вопросы к зачету 1, 4, 7, 10.\\
\cline{4-6}
  &
  &
  & Умеет самостоятельно решать задачи проектирования, моделирования, синтеза и анализа компьютерных моделей технических систем, выбирать и развертывать адекватные компоненты комплекса средств компьютерного автоматизированного проектирования в соответствии с принятыми требованиями к эффективности и приводить формальное обоснование принятых решений документально в соответствии с отраслевыми, государственными и международными спецификациями.
  & Собеседование (УО-1); дискуссия (УО-4); конспект (ПР-7).
  & Вопросы к зачету 1, 4, 7, 10.\\
\cline{4-6}
  &
  &
  & Владеет навыками обоснования, синтеза и анализа проектных решений в области проектирования технических систем с помощью средств компьютерного автоматизированного проектирования.
  & Собеседование (УО-1); дискуссия (УО-4); конспект (ПР-7).
  & Вопросы к зачету 1, 4, 7, 10.\\
\cline{3-6}
  &
  & ОПК-2.2 Знает методы решения задач управления в технических системах
  & Демонстрирует знание основных положений и формальных принципов задач управления в детерминированных и недетерминированных системах, а также методы их реализации в контексте функционального и структурного подходов к блочно иерархическому проектированию.
  & Собеседование (УО-1); дискуссия (УО-4); конспект (ПР-7).
  & Вопросы к зачету 2, 5, 8.\\
\cline{4-6}
  &
  &
  & Умеет применять методы математического формализма для моделирования задач управления в технических системах.
  & Собеседование (УО-1); дискуссия (УО-4); конспект (ПР-7).
  & Вопросы к зачету 2, 5, 8.\\
\cline{4-6}
  &
  &
  & Владеет методами формального анализа, обоснования (квази-) оптимальных технических систем на основе полиномиального представления функций управления и принятия решений.
  & Собеседование (УО-1); дискуссия (УО-4); конспект (ПР-7).
  & Вопросы к зачету 2, 5, 8.\\
\cline{3-6}
  &
  &
  ОПК-4.2 Систематизирует современные математические методы для разработки критериев систем управления в области инновационной деятельности & Знает методы обоснования критериев эффективности технической системы, основные показатели функциональной и нефункциональной эффективности на примерах ГОСТ ИСО Р 9000-2015, ГОСТ Р 57330-2016 и ГОСТ Р ИСО/МЭК 25010-2015, знает аспекты реализации имитационного моделирования технических систем и его ограничения.
  & Собеседование (УО-1); дискуссия (УО-4); конспект (ПР-7).
  & Вопросы к зачету 3, 6, 9.\\
\cline{4-6}
  &
  &
  & Умеет прогнозно оценивать эффективность системы, используя аналитические и численные модели технической системы, включая методы вероятностной оценки загрузки системы и ее функционирования в нестационарном режиме. & Собеседование (УО-1); дискуссия (УО-4); конспект (ПР-7).
  & Вопросы к зачету 3, 6, 9.\\
\cline{4-6}
  &
  &
  & Владеет методами аналитического и численного обоснования критерия эффективности системы, а также оптимизации путем выполнения, в том числе автоматизированного, функционального анализа критерия эффективности применительно к основной и вспомогательной функциям системы.
  & Собеседование (УО-1); дискуссия (УО-4); конспект (ПР-7).
  & Вопросы к зачету 3, 6, 9.\\
\cline{3-6}
  &
  & ОПК-4.3 Вырабатывает и реализует управленческие решения в области инновационной деятельности
  & Знает требования к документации на технические сети и системы, а также на технические средства сопровождения автоматизированного проектирования, регламентирующие государственные и международные стандарты; методы автоматизированной генерации, хранения, сопровождения документации с помощью существующих систем автоматизированного проектирования и программных средств автоматизированной генерации, обработки, хранения и доступа к документации.
  & Собеседование (УО-1); дискуссия (УО-4); конспект (ПР-7).
  & Вопросы к зачету 1, 4, 7, 10.\\
\cline{4-6}
  &
  &
  & Умеет применять средства поиска нормативной документации, воспринимать содержание нормативных спецификаций. Умеет применять средства разметки для автоматизированной генерации, представления, хранения, предоставления доступа и сопровождения нормативной документации к техническим программным и аппаратным решениям при разработке и анализе технических систем.
  & Собеседование (УО-1); дискуссия (УО-4); конспект (ПР-7).
  & Вопросы к зачету 1, 4, 7, 10.\\
\cline{4-6}
  &
  &
  & Владеет навыками использования средств автоматизированной генерации, обработки, хранения, сопровождения и доступа к документации к техническим системам, реализуемым и применяемым при разработке и поддержке телекоммуникационных систем и сетей; составления нормативной документации к системам автоматизированного проектирования технических систем в соответствии с отраслевыми, государственными и международными спецификациями.
  & Собеседование (УО-1); дискуссия (УО-4); конспект (ПР-7).
  & Вопросы к зачету 1, 4, 7, 10.\\
\hline
\end{longtable}
\renewcommand\theadfont{\normalsize\bfseries}
\normalsize

