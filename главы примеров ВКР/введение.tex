\chapter*{Введение}
Быстрое преобразование Фурье (FFT) является эффективной реализацией дискретного преобразования Фурье (DFT).
Из всех дискретных преобразований ДПФ наиболее широко используется в цифровой обработке сигналов \cite{ft-spectroscopy-techreport, infrared-spectroscopy-gost, cosmic-materials-gost, cosmic-materials-iso}.
DFT отображает последовательность дискретных данных в частотную область \cite{ComputerAlgorithms}.
Многие из его свойств аналогичны свойствам преобразования Фурье аналогового сигнала.
Первоначальная разработка ДПФ была выполнена математиками Кули и Туки, за которой последовали различные усовершенствования / модификации другими исследователями (некоторые из них адаптированы к конкретному программному обеспечению / оборудованию).
Независимо от подхода Кули–Туки, несколько алгоритмов, таких как простой множитель, разделенный радиус, векторный радиус, были разработаны основы расщепления вектора и алгоритм преобразования Фурье Винограда (WFTA).
Ряд компаний предоставляют программное обеспечение для реализации БПФ и связанных с ним базовых приложений, таких как свертка / корреляция, фильтрация, спектральный анализ и т.д. на различных платформах.
Также микросхемы DSP общего назначения могут быть запрограммированы для реализации БПФ и других дискретных преобразований.

Одно из применений ДПФ заключается в быстром выполнении сверточных операций над цифровыми данными, используя теорему о свертке, когда сложность свертки во временной области, снижается до линейной сложности эквивалентного поточечного умножения в частотной области.

Примером использования теоремы о свертке является использование теоретико-числового преобразования Фурье над элементами целочисленных колец для реализации наиболее быстрых сегодня известных алгоритмов длинного целочисленного умножения \cite{KaratsubaForHomomorphicEncryption}, которое используется в различных приложениях высокоточного вычисления, в том числе с плавающей точкой, асимметричной криптографии, помехоустойчивого кодирования.

Поэтому \textbf{целью} работы является реализация и формальное обоснование векторной формы алгоритма Шёнхаге-Штрассена для длинного умножения целочисленных данных.

Для достижения данной цели решались следующие \textbf{задачи}.
\begin{enumerate}[wide]
\item Изучить современную научно-техническую литературу по проблеме применения и высокопроизводительной реализации алгоритмов преобразования Фурье над целыми числами, арифметики произвольных конечных целочисленных колец, применения и эффективной реализации сверточных операций.
\item Спроектировать алгоритмы дискретного и быстрого теоретико-числовых преобразований и их использование для быстрого решения задач расчета свертки в целых числах и длинного умножения.
\item Выполнить векторную реализацию свертки над целочисленными кольцами и длинного умножения, используя векторный процессор с поддержкой AVX-512.
\item Выполнить экспериментальное измерение эффективности векторного теоретико-числового, сверточного преобразований над кольцами, а также длинного умножения целых.
\end{enumerate}

